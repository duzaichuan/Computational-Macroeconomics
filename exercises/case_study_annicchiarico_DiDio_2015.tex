\section[Case Study Annicchiarico and Di Dio (2015): Environmental policy and macroeconomic dynamics in a new Keynesian model]{Case Study Annicchiarico and Di Dio (2015): Environmental policy and macroeconomic dynamics in a new Keynesian model\label{ex:CaseStudy.Annicchiarico.DiDio.2015}}

\begin{enumerate}
\item Read the paper by \textcite{Annicchiarico.DiDio_2015_EnvironmentalPolicyMacroeconomic},
  its Online Appendix, and its Erratum available from \url{http://www.barbarannicchiarico.eu/research.htm}.
How does the model differ from the New Keynesian model that we studied in the course so far?

\item Consider the following Dynare files given in the appendix:
\begin{itemize}
  \item \texttt{ann\_dio\_2015\_\_symdecls.inc}
  \item \texttt{ann\_dio\_2015\_\_modeleqs.inc}
  \item \texttt{ann\_dio\_2015\_tbl1\_tbl2\_no\_policy.mod}
  \item \texttt{ann\_dio\_2015\_tbl2\_environmental\_policy.mod}
  \item \texttt{ann\_dio\_2015\_steadystate\_helper\.m}
\end{itemize}
Explain what each file is doing.

\item Note that \texttt{\_modeleqs.inc} misses two equations, one for the abatement effort (A.26), and one describing the environmental policy regime.
However, it contains a \texttt{@\#include "\_environmental\_regime.inc"}  directive.
Create a file with this name and add the missing equations, differentiating the different policy regimes by using Dynare's MACRO preprocessing variables
  and the \texttt{@\#ifdef} directive. That is, the file should look like this:

{\footnotesize
\begin{lstlisting}[style=Matlab-editor,basicstyle=\mlttfamily\scriptsize]
@#ifdef NO_POLICY
  [name='A(26) abatement effort']
  
@#else
  [name='A(26) abatement effort']
  
@#endif

@#ifdef NO_POLICY
  [name='environmental policy regime: no policy (p.6 bottom)']
  
@#endif

@#ifdef CAP_AND_TRADE
  [name='environmental policy regime: cap-and-trade (p.6 bottom)']
  
@#endif

@#ifdef INTENSITY_TARGET
  [name='environmental policy regime: intensity target (p.7 top)']
  
@#endif

@#ifdef TAX_POLICY
  [name='environmental policy regime: tax policy (p.7 top)']
  
@#endif
\end{lstlisting}
}
\item See the hints below and replicate Table 1 and Table 2 of the paper.

\item See the hints below and replicate Figures 1, 2, and 3 of the paper.

\item See the hints below and replicate Tables 3, 4, and 5 of the paper.

\end{enumerate}

\newpage
\paragraph{Hints}

\begin{itemize}

\item Dynare's \texttt{@\#include "SOMEMODILE"} directive is very useful.
  as it enables you to split mod files into several modular components,
  such as different variable declarations, model equations, shocks declarations, calibrations and simulations commands.
In this course, we usually use .inc as the ending for the files,
  to make explicit that these files need to be included in to a mod file as it makes no sense to run them separately.
The main advantage is that you don't have to copy/paste the whole model (at the beginning) or changes to the model (during development).

\item Read about \texttt{save\_params\_and\_steady\_state} and \texttt{load\_params\_and\_steady\_state} in the manual of Dynare.

\item The calibration used in the figures and tables is slightly different.
\begin{itemize}
  \item set $\phi_1$ to $0.180$ instead of $0.185$.
  \item set the government spending ratio in the no-policy regime to $0.102158273389097$.
  \item set the emissions share to total emissions equal to $0.20840121$ in the no-policy regime.
  \item set $\sigma_a=1$ and $\sigma_g=1$ for figures 1 and 2.
  \item set $\sigma_\eta=0.0025$ and $\iota_\pi=1.5$ for figure 3.
\end{itemize} 
  
\item The figures with the IRFs are computed from a first-order approximation.
Something is wrong in the published figures at the impact period,
  see the Erratum that explains that there was a mistake in the code
  and provides the correctly generated IRFs (except for figure 2b).

\item The moments in Tables 3 to 6 are computed from a second-order approximation with pruning.
Note that in more recent versions of Dynare,
  the moments are computed based on closed-form expressions from the pruned state-space system
  instead of using second-order accurate expressions.
So tiny errors in the unconditional means and somewhat different variances and autocorrelations are to be expected.

\end{itemize}


\begin{solution}\textbf{Solution to \nameref{ex:CaseStudy.Annicchiarico.DiDio.2015}}
\ifDisplaySolutions
\input{exercises/case_study_annicchiarico_DiDio_2015_solution.tex}
\fi
\newpage
\end{solution}